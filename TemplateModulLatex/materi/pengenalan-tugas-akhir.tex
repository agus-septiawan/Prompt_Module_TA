\begin{center}
    \addcontentsline{toc}{section}{Modul Pengenalan Tugas Akhir Informatika}
    \textbf{Modul Pengenalan Tugas Akhir Informatika}

    \textbf{Panduan Lengkap Penulisan Tugas Akhir untuk Mahasiswa Informatika}
\end{center}

\addcontentsline{toc}{subsection}{Deskripsi Materi}
\section*{Deskripsi Singkat}
Modul ini memberikan panduan komprehensif tentang penulisan Tugas Akhir (TA) untuk mahasiswa Jurusan Informatika. Materi mencakup seluruh aspek mulai dari pemahaman konsep dasar, struktur penulisan, metodologi penelitian, hingga tips praktis untuk menghasilkan karya ilmiah yang berkualitas tinggi. Modul ini dirancang khusus untuk membantu mahasiswa memahami setiap detail penulisan TA dengan mudah dan sistematis.

\addcontentsline{toc}{subsection}{Tujuan Pembelajaran}
\section*{Tujuan Pembelajaran}
Setelah mempelajari modul ini, mahasiswa diharapkan mampu:
\begin{enumerate}
    \item Memahami konsep dasar dan tujuan Tugas Akhir dalam konteks pendidikan Informatika
    \item Menguasai struktur penulisan TA sesuai standar akademik Universitas Syiah Kuala
    \item Memahami perbedaan penelitian kualitatif dan kuantitatif dalam bidang Informatika
    \item Merancang metodologi penelitian yang tepat untuk bidang data science
    \item Menentukan judul penelitian yang baik, fokus, dan sesuai scope
    \item Memahami cara penulisan setiap bab dengan benar dan sistematis
    \item Menguasai teknik sitasi dan referensi yang proper menggunakan format yang benar
    \item Memahami kapan dan bagaimana menggunakan gambar, tabel, dan kode dalam TA
    \item Menyusun workflow penelitian dari awal hingga akhir
    \item Menganalisis contoh TA yang baik dan mengidentifikasi best practices
\end{enumerate}

\addcontentsline{toc}{subsection}{Bagian 1 - Pengertian dan Konsep Dasar Tugas Akhir}
\subsection*{Bagian 1 - Pengertian dan Konsep Dasar Tugas Akhir}

\textbf{Apa itu Tugas Akhir?}

Tugas Akhir (TA) adalah karya ilmiah yang wajib diselesaikan oleh mahasiswa sebagai syarat untuk memperoleh gelar Sarjana Komputer. TA merupakan bukti kemampuan mahasiswa dalam:

\begin{enumerate}
    \item \textbf{Mengidentifikasi masalah}: Menemukan permasalahan yang relevan dalam bidang Informatika
    \item \textbf{Merancang solusi}: Mengembangkan pendekatan atau metode untuk menyelesaikan masalah
    \item \textbf{Mengimplementasikan}: Mewujudkan solusi dalam bentuk sistem, algoritma, atau aplikasi
    \item \textbf{Mengevaluasi}: Mengukur dan menganalisis efektivitas solusi yang dibuat
    \item \textbf{Mengkomunikasikan}: Menyajikan hasil penelitian dalam bentuk tulisan ilmiah
\end{enumerate}

\textbf{Mengapa Tugas Akhir Penting?}

\begin{enumerate}
    \item \textbf{Pengembangan Kemampuan Riset}: TA melatih mahasiswa untuk berpikir sistematis dan analitis
    \item \textbf{Penerapan Ilmu}: Mengaplikasikan teori yang dipelajari selama kuliah ke masalah nyata
    \item \textbf{Kontribusi Ilmiah}: Memberikan sumbangan pengetahuan baru atau perbaikan pada bidang Informatika
    \item \textbf{Persiapan Karir}: Membekali mahasiswa dengan pengalaman penelitian untuk dunia kerja atau studi lanjut
    \item \textbf{Validasi Kompetensi}: Membuktikan bahwa mahasiswa telah menguasai ilmu Informatika secara komprehensif
\end{enumerate}

\textbf{Karakteristik TA Informatika}

TA Informatika memiliki ciri khas yang membedakannya dari bidang lain:
\begin{itemize}
    \item \textbf{Berbasis Teknologi}: Menggunakan teknologi informasi sebagai alat atau objek penelitian
    \item \textbf{Problem-Solving Oriented}: Fokus pada penyelesaian masalah praktis yang ada di masyarakat
    \item \textbf{Implementatif}: Biasanya menghasilkan produk berupa sistem, aplikasi, atau algoritma
    \item \textbf{Data-Driven}: Menggunakan data sebagai dasar analisis dan evaluasi
    \item \textbf{Measurable}: Hasil dapat diukur secara objektif menggunakan metrik tertentu
    \item \textbf{Reproducible}: Penelitian dapat diulang oleh peneliti lain dengan hasil yang konsisten
\end{itemize}

\addcontentsline{toc}{subsection}{Bagian 2 - Jenis Penelitian dalam Informatika}
\subsection*{Bagian 2 - Jenis Penelitian dalam Informatika}

\textbf{Penelitian Kuantitatif vs Kualitatif dalam Informatika}

Pemahaman tentang jenis penelitian sangat penting untuk menentukan pendekatan yang tepat dalam TA Anda.

\textbf{1. Penelitian Kuantitatif}

Penelitian kuantitatif dalam Informatika menggunakan data numerik dan metode statistik untuk menganalisis fenomena.

\textbf{Karakteristik Penelitian Kuantitatif:}
\begin{itemize}
    \item \textbf{Data Numerik}: Menggunakan angka, metrik, dan pengukuran yang dapat dikuantifikasi
    \item \textbf{Objektif}: Hasil dapat diverifikasi dan direproduksi oleh peneliti lain
    \item \textbf{Statistik}: Menggunakan analisis statistik untuk interpretasi data
    \item \textbf{Generalisasi}: Hasil dapat digeneralisasi ke populasi yang lebih luas
    \item \textbf{Hipotesis}: Biasanya menggunakan hipotesis yang dapat diuji secara statistik
    \item \textbf{Deduktif}: Berangkat dari teori untuk diuji dengan data
\end{itemize}

\textbf{Contoh Data Kuantitatif dalam Informatika:}
\begin{itemize}
    \item \textbf{Metrik Performa Model}: Akurasi (85.5\%, 92.3\%), Precision, Recall, F1-Score, BLEU Score
    \item \textbf{Metrik Waktu}: Waktu eksekusi algoritma (0.25 detik, 1.5 menit), response time, latency
    \item \textbf{Metrik Sistem}: Throughput (1000 request/detik), CPU usage (\%), memory consumption (MB)
    \item \textbf{Metrik Storage}: Ukuran file (2.5 MB, 512 KB), database size, compression ratio
    \item \textbf{Metrik Pengguna}: Jumlah pengguna, transaksi, click-through rate, conversion rate
    \item \textbf{Metrik Kualitas}: Error rate, availability (99.9\%), scalability metrics
\end{itemize}

\textbf{Contoh Judul Penelitian Kuantitatif:}
\begin{itemize}
    \item "Perbandingan Akurasi Algoritma CNN dan SVM untuk Klasifikasi Citra"
    \item "Analisis Performa Algoritma Clustering K-Means vs DBSCAN pada Dataset E-Commerce"
    \item "Optimasi Kecepatan Query Database menggunakan Indexing Berbasis B-Tree"
    \item "Evaluasi Efisiensi Algoritma Sorting pada Big Data menggunakan MapReduce"
\end{itemize}

\textbf{Contoh Tipe Data Kuantitatif:}
\begin{lstlisting}[language=python, style=python, caption=Contoh Data Kuantitatif dalam Penelitian]
# Contoh data hasil eksperimen machine learning
experiment_results = {
    'model_name': ['CNN', 'SVM', 'Random Forest', 'Neural Network'],
    'accuracy': [0.855, 0.923, 0.891, 0.876],
    'precision': [0.834, 0.912, 0.878, 0.865],
    'recall': [0.867, 0.934, 0.905, 0.887],
    'f1_score': [0.850, 0.923, 0.891, 0.876],
    'training_time': [120.5, 95.3, 110.8, 105.2],  # dalam detik
    'model_size': [15.2, 8.7, 12.1, 10.5],         # dalam MB
    'inference_time': [0.025, 0.018, 0.022, 0.020] # dalam detik
}

# Data perbandingan algoritma
algorithm_performance = {
    'Dataset Size': [1000, 5000, 10000, 50000, 100000],
    'Algorithm A Time': [0.1, 0.5, 1.2, 6.8, 15.2],
    'Algorithm B Time': [0.2, 0.8, 2.1, 12.5, 28.9],
    'Algorithm A Memory': [10, 25, 50, 180, 350],
    'Algorithm B Memory': [15, 35, 70, 220, 420]
}
\end{lstlisting}

\textbf{2. Penelitian Kualitatif}

Penelitian kualitatif dalam Informatika fokus pada pemahaman mendalam tentang fenomena, proses, atau pengalaman pengguna.

\textbf{Karakteristik Penelitian Kualitatif:}
\begin{itemize}
    \item \textbf{Data Deskriptif}: Menggunakan teks, gambar, audio, atau observasi
    \item \textbf{Subjektif}: Melibatkan interpretasi dan pemahaman kontekstual
    \item \textbf{Eksploratori}: Menggali pemahaman mendalam tentang fenomena
    \item \textbf{Kontekstual}: Hasil terikat pada konteks spesifik penelitian
    \item \textbf{Fleksibel}: Metodologi dapat disesuaikan selama penelitian berlangsung
    \item \textbf{Induktif}: Membangun teori dari data yang dikumpulkan
\end{itemize}

\textbf{Contoh Data Kualitatif dalam Informatika:}
\begin{itemize}
    \item \textbf{User Experience}: Feedback pengguna tentang kemudahan penggunaan aplikasi
    \item \textbf{Analisis Kebutuhan}: Hasil wawancara dengan stakeholder tentang kebutuhan sistem
    \item \textbf{Observasi Perilaku}: Pengamatan cara pengguna berinteraksi dengan interface
    \item \textbf{Code Review}: Analisis kualitas kode untuk memahami pola programming
    \item \textbf{Studi Kasus}: Implementasi sistem di organisasi tertentu
    \item \textbf{Content Analysis}: Analisis konten media sosial, log sistem, atau dokumen
    \item \textbf{Expert Opinion}: Pendapat ahli tentang teknologi atau metodologi
\end{itemize}

\textbf{Contoh Judul Penelitian Kualitatif:}
\begin{itemize}
    \item "Analisis User Experience pada Aplikasi E-Learning: Studi Kasus Universitas XYZ"
    \item "Studi Etnografi Penggunaan Media Sosial di Kalangan Remaja"
    \item "Analisis Kebutuhan Sistem Informasi Manajemen Rumah Sakit: Pendekatan Participatory Design"
    \item "Eksplorasi Faktor-Faktor yang Mempengaruhi Adopsi Teknologi Cloud Computing di UKM"
\end{itemize}

\textbf{Contoh Tipe Data Kualitatif:}
\begin{lstlisting}[language=python, style=python, caption=Contoh Data Kualitatif dalam Penelitian]
# Contoh data hasil wawancara pengguna
user_interviews = [
    {
        'participant_id': 'P001',
        'age_group': '20-25',
        'experience': 'Pemula',
        'feedback': '''Aplikasi mudah digunakan untuk fitur dasar,
                      tetapi loading terlalu lama saat upload file besar.
                      Interface cukup intuitif tapi butuh tutorial.''',
        'pain_points': ['Loading lambat', 'Kurang tutorial', 'Error message tidak jelas'],
        'suggestions': ['Perbaiki kecepatan', 'Tambah help system', 'Improve error handling']
    },
    {
        'participant_id': 'P002',
        'age_group': '26-30',
        'experience': 'Mahir',
        'feedback': '''Interface sangat user-friendly dan fitur lengkap.
                      Saya suka dengan shortcut keyboard yang tersedia.
                      Perlu ditambahkan fitur batch processing.''',
        'pain_points': ['Tidak ada batch processing', 'Export format terbatas'],
        'suggestions': ['Tambah batch processing', 'More export options']
    }
]

# Contoh hasil observasi perilaku pengguna
observation_notes = """
OBSERVASI SESI 1 (Tanggal: 15 Januari 2024)
Durasi: 30 menit
Partisipan: 5 orang mahasiswa

Temuan Utama:
1. 80% pengguna menggunakan shortcut keyboard untuk navigasi cepat
2. Pengguna cenderung skip tutorial dan langsung mencoba fitur
3. Rata-rata 3 kali trial-error sebelum berhasil menyelesaikan task
4. Pengguna sering bingung dengan icon yang tidak familiar
5. Loading indicator sangat membantu mengurangi anxiety pengguna

Quotes Penting:
- "Kenapa tombol save-nya tidak ada di tempat yang biasa?"
- "Loading-nya lama banget, saya kira aplikasinya hang"
- "Interface-nya bagus tapi kurang intuitif untuk pemula"
"""
\end{lstlisting}

\textbf{3. Penelitian Mixed Methods (Campuran)}

Banyak TA Informatika, terutama di bidang data science, menggunakan pendekatan campuran:
\begin{itemize}
    \item \textbf{Kuantitatif}: Untuk mengukur performa teknis sistem (akurasi, kecepatan, efisiensi)
    \item \textbf{Kualitatif}: Untuk memahami kebutuhan pengguna, konteks penggunaan, dan interpretasi hasil
\end{itemize}

\textbf{Contoh Penelitian Mixed Methods:}
"Pengembangan Sistem Rekomendasi E-Commerce menggunakan Collaborative Filtering: Analisis Performa dan User Experience"
\begin{itemize}
    \item \textbf{Kuantitatif}: Mengukur akurasi rekomendasi, response time, click-through rate
    \item \textbf{Kualitatif}: Wawancara pengguna tentang kepuasan dan kegunaan rekomendasi
\end{itemize}

\addcontentsline{toc}{subsection}{Bagian 3 - Struktur Lengkap Penulisan Tugas Akhir}
\subsection*{Bagian 3 - Struktur Lengkap Penulisan Tugas Akhir}

Berdasarkan template yang Anda berikan, struktur TA Informatika terdiri dari bagian-bagian berikut:

\textbf{BAGIAN AWAL (Front Matter)}

\begin{enumerate}
    \item \textbf{Halaman Judul (Cover)}
    \begin{itemize}
        \item \textbf{Isi}: Judul penelitian, nama mahasiswa, NPM, program studi, fakultas, universitas, tahun
        \item \textbf{Fungsi}: Identitas resmi dokumen TA
        \item \textbf{Format}: Menggunakan huruf kapital untuk elemen penting
        \item \textbf{Contoh}: Lihat template - judul ditulis dengan font besar dan bold
    \end{itemize}

    \item \textbf{Halaman Pengesahan}
    \begin{itemize}
        \item \textbf{Isi}: Persetujuan pembimbing I, pembimbing II, koordinator prodi, dan dekan
        \item \textbf{Fungsi}: Validasi akademik bahwa TA telah memenuhi standar
        \item \textbf{Catatan}: Berisi tanda tangan dan cap basah, tanggal lulus sidang
    \end{itemize}

    \item \textbf{Pernyataan Bebas Plagiasi}
    \begin{itemize}
        \item \textbf{Isi}: Pernyataan bahwa karya adalah hasil sendiri dan bebas plagiasi
        \item \textbf{Fungsi}: Jaminan integritas akademik
        \item \textbf{Penting}: Harus ditandatangani di atas materai 10.000
    \end{itemize}

    \item \textbf{Surat Pernyataan}
    \begin{itemize}
        \item \textbf{Isi}: Pernyataan tentang publikasi atau non-publikasi hasil penelitian
        \item \textbf{Fungsi}: Mengatur hak publikasi dan distribusi karya
        \item \textbf{Durasi}: Biasanya 5 tahun dari tanggal kelulusan
    \end{itemize}

    \item \textbf{Abstrak Indonesia}
    \begin{itemize}
        \item \textbf{Panjang}: Maksimal 250 kata
        \item \textbf{Isi}: Ringkasan lengkap penelitian (masalah, metode, hasil, kesimpulan)
        \item \textbf{Kata Kunci}: 3-5 kata kunci yang relevan dengan penelitian
        \item \textbf{Bahasa}: Formal, padat, dan informatif
    \end{itemize}

    \item \textbf{Abstract English}
    \begin{itemize}
        \item \textbf{Fungsi}: Versi bahasa Inggris dari abstrak Indonesia
        \item \textbf{Penting}: Harus konsisten dengan abstrak Indonesia
        \item \textbf{Keywords}: Terjemahan kata kunci dalam bahasa Inggris
    \end{itemize}
\end{enumerate}

\textbf{BAGIAN INTI (Main Content)}

\textbf{BAB I - PENDAHULUAN}

Bab ini adalah "pintu masuk" penelitian Anda. Fungsinya untuk memperkenalkan masalah dan meyakinkan pembaca bahwa penelitian ini penting.

\textbf{Struktur BAB I:}

\textbf{1.1 Latar Belakang}
\begin{itemize}
    \item \textbf{Panjang}: 2-3 halaman
    \item \textbf{Pola Penulisan}: Umum → Khusus → Spesifik (Funnel Approach)
    \item \textbf{Isi}: Konteks masalah, mengapa masalah ini muncul, mengapa perlu diselesaikan
    \item \textbf{Sitasi}: Minimal 10-15 referensi untuk mendukung argumen
\end{itemize}

\textbf{Pola Penulisan Latar Belakang:}
\begin{lstlisting}[language=bash, style=bash, caption=Struktur Latar Belakang]
Paragraf 1: Konteks umum bidang penelitian
"Citra medis, seperti MRI, X-Ray, CT-Scan menjadi fokus penelitian
utama dalam dunia medis [sitasi]..."

Paragraf 2: Masalah yang ada saat ini
"Kesalahan analisis citra medis oleh tenaga medis terkadang bisa
terjadi karena faktor manusia [sitasi]..."

Paragraf 3: Dampak masalah dan urgensi
"Ketika manusia melakukan pekerjaan repetitif, dapat menyebabkan
kelelahan dan berkurangnya konsentrasi [sitasi]..."

Paragraf 4: Solusi yang diusulkan
"Dibutuhkan sebuah sistem yang dapat membantu tenaga medis dalam
menjawab permasalahan pada citra medis [sitasi]..."

Paragraf 5: Pendekatan penelitian spesifik
"Penelitian ini bertujuan untuk mengembangkan sistem VQA yang
memanfaatkan teknik transfer learning [sitasi]..."
\end{lstlisting}

\textbf{1.2 Rumusan Masalah}
\begin{itemize}
    \item \textbf{Jumlah}: 3-5 pertanyaan penelitian
    \item \textbf{Format}: Menggunakan enumerate, dimulai dengan kata tanya
    \item \textbf{Karakteristik}: Spesifik, terukur, dapat dijawab melalui penelitian
    \item \textbf{Urutan}: Dari yang paling fundamental ke yang paling spesifik
\end{itemize}

\textbf{Contoh Rumusan Masalah yang Baik:}
\begin{enumerate}
    \item Bagaimana membangun model VQA menggunakan citra medis dengan menerapkan teknik transfer learning?
    \item Bagaimana membandingkan performa model VQA pada arsitektur VGG19-LSTM dan BLIP untuk dataset PathVQA dan VQA-RAD?
    \item Bagaimana menerapkan model VQA terbaik untuk menjawab pertanyaan berdasarkan citra medis?
\end{enumerate}

\textbf{1.3 Tujuan Penelitian}
\begin{itemize}
    \item \textbf{Hubungan}: Satu-satu dengan rumusan masalah
    \item \textbf{Format}: Menggunakan kata kerja aktif (membangun, menganalisis, mengimplementasikan)
    \item \textbf{Spesifik}: Harus jelas dan terukur
\end{itemize}

\textbf{1.4 Manfaat Penelitian}
\begin{itemize}
    \item \textbf{Manfaat Teoritis}: Kontribusi terhadap ilmu pengetahuan
    \item \textbf{Manfaat Praktis}: Kegunaan bagi industri atau masyarakat
    \item \textbf{Manfaat Metodologis}: Pengembangan metode atau teknik baru
\end{itemize}

\textbf{BAB II - TINJAUAN KEPUSTAKAAN}

Bab ini adalah "fondasi teoritis" penelitian. Fungsinya:
\begin{itemize}
    \item Menunjukkan pemahaman mendalam tentang bidang penelitian
    \item Mengidentifikasi gap penelitian yang akan diisi
    \item Memberikan dasar teoritis untuk metodologi yang dipilih
\end{itemize}

\textbf{Struktur BAB II (Berdasarkan Template):}

\textbf{2.1 Citra Medis}
\begin{itemize}
    \item \textbf{Isi}: Definisi, jenis-jenis, karakteristik citra medis
    \item \textbf{Fungsi}: Memberikan pemahaman dasar tentang domain penelitian
\end{itemize}

\textbf{2.2 Deep Learning}
\begin{itemize}
    \item \textbf{Isi}: Konsep dasar, kategori pendekatan, artificial neural network
    \item \textbf{Subfungsi}: Fungsi aktivasi, fungsi loss, fungsi optimasi
    \item \textbf{Gambar}: Diagram arsitektur neural network
\end{itemize}

\textbf{2.3 Convolutional Neural Network}
\begin{itemize}
    \item \textbf{Isi}: Arsitektur CNN, cara kerja konvolusi, pooling
    \item \textbf{Gambar}: Diagram arsitektur CNN dengan penjelasan setiap layer
\end{itemize}

\textbf{2.4 Long Short Term Memory}
\begin{itemize}
    \item \textbf{Isi}: Arsitektur LSTM, cara mengatasi vanishing gradient
    \item \textbf{Gambar}: Diagram internal LSTM cell
\end{itemize}

\textbf{2.5 Transfer Learning}
\begin{itemize}
    \item \textbf{Isi}: Konsep, keuntungan, implementasi
    \item \textbf{Gambar}: Diagram konsep transfer learning
\end{itemize}

\textbf{2.6 Matriks Evaluasi}
\begin{itemize}
    \item \textbf{Isi}: Akurasi, BLEU score, confusion matrix
    \item \textbf{Rumus}: Formula matematika untuk setiap metrik
\end{itemize}

\textbf{2.7 Visual Question Answering}
\begin{itemize}
    \item \textbf{Isi}: Definisi VQA, arsitektur model, computer vision, NLP
    \item \textbf{Gambar}: Diagram arsitektur VQA
\end{itemize}

\textbf{2.8 Medical Visual Question Answering Dataset}
\begin{itemize}
    \item \textbf{Isi}: PathVQA dataset, VQA-RAD dataset
    \item \textbf{Detail}: Karakteristik, jumlah data, format
\end{itemize}

\textbf{2.9 Penelitian Terkait}
\begin{itemize}
    \item \textbf{Isi}: Review penelitian sebelumnya yang relevan
    \item \textbf{Fungsi}: Menunjukkan gap dan positioning penelitian
\end{itemize}

\textbf{BAB III - METODOLOGI PENELITIAN}

Bab ini menjelaskan "bagaimana" penelitian dilakukan secara detail.

\textbf{Struktur BAB III (Berdasarkan Template):}

\textbf{3.1 Waktu dan Lokasi Penelitian}
\begin{itemize}
    \item \textbf{Isi}: Tempat penelitian, durasi penelitian
    \item \textbf{Contoh}: "Ruang Lab Sistem Informasi dan Database, 4 bulan (Februari-Mei 2024)"
\end{itemize}

\textbf{3.2 Jadwal Pelaksanaan}
\begin{itemize}
    \item \textbf{Format}: Tabel Gantt chart
    \item \textbf{Nama Tabel}: "Jadwal pelaksanaan penelitian"
    \item \textbf{Isi}: Timeline setiap aktivitas penelitian
\end{itemize}

\textbf{3.3 Alat dan Bahan}
\begin{itemize}
    \item \textbf{3.3.1 Perangkat Keras}: Spesifikasi komputer, server, GPU
    \item \textbf{3.3.2 Perangkat Lunak}: OS, IDE, framework, library
    \item \textbf{Detail}: Versi spesifik setiap software yang digunakan
\end{itemize}

\textbf{3.4 Metode Penelitian}
\begin{itemize}
    \item \textbf{Diagram Alir}: Flowchart metodologi penelitian
    \item \textbf{Tahapan}: Identifikasi masalah → Studi literatur → Pengumpulan data → dst.
\end{itemize}

\textbf{Nama-nama Tabel Penting di BAB III:}
\begin{itemize}
    \item \textbf{Tabel 3.1}: "Jadwal pelaksanaan penelitian"
    \item \textbf{Tabel 3.2}: "Proses case folding" (untuk preprocessing text)
    \item \textbf{Tabel 3.3}: "Proses remove punctuation"
    \item \textbf{Tabel 3.4}: "Proses stemming"
    \item \textbf{Tabel 3.5}: "Proses tokenisasi"
\end{itemize}

\textbf{BAB IV - HASIL DAN PEMBAHASAN}

Bab ini menyajikan hasil penelitian dan analisisnya.

\textbf{Struktur BAB IV (Berdasarkan Template):}

\textbf{4.1 Pengumpulan Data}
\begin{itemize}
    \item \textbf{Isi}: Deskripsi dataset yang digunakan
    \item \textbf{Gambar}: Sampel data dari dataset
    \item \textbf{Tabel}: Statistik dataset (jumlah data, distribusi kelas, dll)
\end{itemize}

\textbf{4.2 Pengembangan Model}
\begin{itemize}
    \item \textbf{Isi}: Implementasi model yang dibangun
    \item \textbf{Gambar}: Arsitektur model yang dikembangkan
    \item \textbf{Code}: Potongan kode penting (menggunakan lstlisting)
\end{itemize}

\textbf{4.3 Augmentasi Data}
\begin{itemize}
    \item \textbf{Isi}: Teknik augmentasi yang diterapkan
    \item \textbf{Tabel}: Contoh data sebelum dan sesudah augmentasi
\end{itemize}

\textbf{4.4 Analisis Hasil}
\begin{itemize}
    \item \textbf{Tabel}: Hasil eksperimen dengan berbagai konfigurasi
    \item \textbf{Grafik}: Visualisasi perbandingan performa
    \item \textbf{Analisis}: Interpretasi hasil dan pembahasan
\end{itemize}

\textbf{4.5 Pengembangan Sistem}
\begin{itemize}
    \item \textbf{Isi}: Implementasi sistem berbasis web
    \item \textbf{Gambar}: Screenshot interface sistem
    \item \textbf{Tabel}: Spesifikasi API endpoint
\end{itemize}

\textbf{BAB V - KESIMPULAN DAN SARAN}

\textbf{5.1 Kesimpulan}
\begin{itemize}
    \item \textbf{Isi}: Jawaban terhadap rumusan masalah
    \item \textbf{Format}: Enumerate, satu poin untuk setiap rumusan masalah
    \item \textbf{Berdasarkan Data}: Harus didukung oleh hasil eksperimen
\end{itemize}

\textbf{5.2 Saran}
\begin{itemize}
    \item \textbf{Isi}: Rekomendasi untuk penelitian selanjutnya
    \item \textbf{Jenis}: Perbaikan metodologi, eksplorasi dataset baru, pengembangan fitur
\end{itemize}

\addcontentsline{toc}{subsection}{Bagian 4 - Kapan Menggunakan Gambar, Tabel, dan Kode}
\subsection*{Bagian 4 - Kapan Menggunakan Gambar, Tabel, dan Kode}

\textbf{Penggunaan Gambar (Figures)}

\textbf{Kapan Menggunakan Gambar:}
\begin{enumerate}
    \item \textbf{Arsitektur Sistem}: Diagram blok, flowchart, network architecture
    \item \textbf{Hasil Visualisasi}: Grafik performa, plot data, confusion matrix
    \item \textbf{Interface}: Screenshot aplikasi, mockup design
    \item \textbf{Konsep Teoritis}: Diagram konsep, model matematis
    \item \textbf{Workflow}: Diagram alir penelitian, proses bisnis
\end{enumerate}

\textbf{Contoh Gambar dari Template:}
\begin{itemize}
    \item \textbf{Gambar 2.1}: "Kategori pendekatan deep learning"
    \item \textbf{Gambar 2.3}: "Arsitektur CNN"
    \item \textbf{Gambar 2.4}: "Arsitektur LSTM"
    \item \textbf{Gambar 3.1}: "Diagram alir penelitian"
    \item \textbf{Gambar 4.8}: "Halaman utama sistem medis cerdas berbasis web"
\end{itemize}

\textbf{Best Practice Gambar:}
\begin{lstlisting}[language=bash, style=bash, caption=Format Gambar dalam LaTeX]
\begin{figure}[H]
    \centering
    \includegraphics[width=\textwidth, height=7cm]{image/path/filename.png}
    \caption{Deskripsi gambar yang jelas dan informatif}
    \label{fig:label-gambar}
\end{figure}
\end{lstlisting}

\textbf{Penggunaan Tabel (Tables)}

\textbf{Kapan Menggunakan Tabel:}
\begin{enumerate}
    \item \textbf{Data Eksperimen}: Hasil perbandingan model, metrik evaluasi
    \item \textbf{Spesifikasi}: Hardware, software, dataset characteristics
    \item \textbf{Jadwal}: Timeline penelitian, milestone
    \item \textbf{Preprocessing}: Contoh transformasi data
    \item \textbf{Perbandingan}: Comparison matrix antar metode
\end{enumerate}

\textbf{Contoh Tabel dari Template:}
\begin{itemize}
    \item \textbf{Tabel 3.1}: "Jadwal pelaksanaan penelitian"
    \item \textbf{Tabel 4.6}: "Konfigurasi hyperparameter pada awal eksperimen"
    \item \textbf{Tabel 4.7}: "Hasil akurasi pelatihan dengan VGG19-LSTM"
    \item \textbf{Tabel 4.13}: "Spesifikasi model BLIP sebelum dan sesudah dikuantisasi"
\end{itemize}

\textbf{Penggunaan Kode (Code Listings)}

\textbf{Kapan Menggunakan Kode:}
\begin{enumerate}
    \item \textbf{Algoritma Utama}: Implementasi algoritma yang menjadi kontribusi penelitian
    \item \textbf{Preprocessing}: Kode untuk cleaning dan transformasi data
    \item \textbf{Model Architecture}: Definisi model neural network
    \item \textbf{Evaluation}: Kode untuk menghitung metrik evaluasi
    \item \textbf{API Implementation}: Endpoint penting dalam sistem
\end{enumerate}

\textbf{Best Practice Kode dalam TA:}
\begin{enumerate}
    \item \textbf{Gunakan lstlisting}: Jangan screenshot kode
    \item \textbf{Syntax Highlighting}: Gunakan style yang sesuai bahasa pemrograman
    \item \textbf{Caption}: Berikan caption yang menjelaskan fungsi kode
    \item \textbf{Komentar}: Tambahkan komentar untuk bagian penting
    \item \textbf{Panjang}: Maksimal 20-30 baris per listing
\end{enumerate}

\textbf{Contoh Format Kode:}
\begin{lstlisting}[language=python, style=python, caption=Implementasi Model CNN untuk Klasifikasi Citra]
import tensorflow as tf
from tensorflow.keras import layers, models

def create_cnn_model(input_shape, num_classes):
    """
    Membuat model CNN untuk klasifikasi citra

    Args:
        input_shape: Dimensi input (height, width, channels)
        num_classes: Jumlah kelas untuk klasifikasi

    Returns:
        model: Model CNN yang sudah dikompilasi
    """
    model = models.Sequential([
        # Convolutional layers
        layers.Conv2D(32, (3, 3), activation='relu', input_shape=input_shape),
        layers.MaxPooling2D((2, 2)),
        layers.Conv2D(64, (3, 3), activation='relu'),
        layers.MaxPooling2D((2, 2)),
        layers.Conv2D(64, (3, 3), activation='relu'),

        # Dense layers
        layers.Flatten(),
        layers.Dense(64, activation='relu'),
        layers.Dropout(0.5),
        layers.Dense(num_classes, activation='softmax')
    ])

    # Compile model
    model.compile(
        optimizer='adam',
        loss='categorical_crossentropy',
        metrics=['accuracy']
    )

    return model
\end{lstlisting}

\addcontentsline{toc}{subsection}{Bagian 5 - Analisis Detail Template TA: Studi Kasus VQA}
\subsection*{Bagian 5 - Analisis Detail Template TA: Studi Kasus VQA}

Mari kita bedah template TA yang Anda berikan untuk memahami struktur dan logika penulisan yang baik.

\textbf{Analisis Judul Template:}
"Perbandingan Kinerja Arsitektur VGG19-LSTM dan BLIP dalam Visual Question Answering (VQA) pada Citra Medis"

\textbf{Mengapa Judul Ini Baik:}
\begin{enumerate}
    \item \textbf{Spesifik}: Menyebutkan metode yang dibandingkan (VGG19-LSTM vs BLIP)
    \item \textbf{Jelas}: Menjelaskan task yang dilakukan (VQA)
    \item \textbf{Domain}: Menentukan domain aplikasi (citra medis)
    \item \textbf{Measurable}: "Perbandingan kinerja" menunjukkan akan ada evaluasi kuantitatif
    \item \textbf{Panjang}: Tidak terlalu panjang (< 20 kata)
\end{enumerate}

\textbf{Analisis Latar Belakang Template:}

\textbf{Paragraf 1}: Konteks umum
\begin{lstlisting}[language=bash, style=bash, caption=Analisis Struktur Latar Belakang]
"Citra medis, seperti Magnetic Resonance Imaging (MRI), X-Ray,
Ultrasonography (USG)... menjadi fokus penelitian utama dalam dunia medis"

ANALISIS:
- Dimulai dengan konteks yang luas (citra medis secara umum)
- Memberikan contoh konkret (MRI, X-Ray, USG)
- Menggunakan sitasi untuk mendukung pernyataan
- Menetapkan domain penelitian
\end{lstlisting}

\textbf{Paragraf 2}: Identifikasi masalah
\begin{lstlisting}[language=bash, style=bash, caption=Identifikasi Masalah]
"kesalahan analisis citra medis oleh tenaga medis terkadang bisa saja
terjadi karena sifat manusia yang rentan terhadap kesalahan"

ANALISIS:
- Mengidentifikasi masalah spesifik (human error)
- Memberikan alasan mengapa masalah ini terjadi
- Menggunakan sitasi untuk mendukung klaim
\end{lstlisting}

\textbf{Paragraf 3}: Dampak dan urgensi
\begin{lstlisting}[language=bash, style=bash, caption=Dampak Masalah]
"Ketika manusia melakukan pekerjaan yang repetitif, ini dapat menyebabkan
kelelahan dan berkurangnya konsentrasi"

ANALISIS:
- Menjelaskan dampak dari masalah yang diidentifikasi
- Memberikan kontras dengan solusi AI
- Membangun argumen untuk pentingnya penelitian
\end{lstlisting}

\textbf{Paragraf 4}: Solusi yang diusulkan
\begin{lstlisting}[language=bash, style=bash, caption=Solusi yang Diusulkan]
"dibutuhkan sebuah sistem yang dapat membantu tenaga medis dalam
menjawab permasalahan yang terdapat pada citra medis"

ANALISIS:
- Mengusulkan solusi spesifik (sistem VQA)
- Menjelaskan manfaat untuk stakeholder
- Memperkenalkan konsep VQA
\end{lstlisting}

\textbf{Paragraf 5}: Pendekatan penelitian
\begin{lstlisting}[language=bash, style=bash, caption=Pendekatan Penelitian]
"Penelitian ini bertujuan untuk mengembangkan sebuah sistem VQA yang
memanfaatkan dua dataset yaitu PathVQA dan VQA-RAD"

ANALISIS:
- Menjelaskan pendekatan spesifik yang akan digunakan
- Menyebutkan dataset yang akan digunakan
- Menjelaskan teknik yang akan diterapkan (transfer learning)
\end{lstlisting}

\textbf{Analisis Rumusan Masalah Template:}

\begin{enumerate}
    \item "Bagaimana membangun model VQA menggunakan citra medis dengan menerapkan teknik transfer learning?"
    \begin{itemize}
        \item \textbf{Fokus}: Pembangunan model
        \item \textbf{Teknik}: Transfer learning
        \item \textbf{Domain}: Citra medis
    \end{itemize}

    \item "Bagaimana membandingkan performa model VQA pada arsitektur VGG19-LSTM dan BLIP untuk dataset PathVQA dan VQA-RAD?"
    \begin{itemize}
        \item \textbf{Fokus}: Perbandingan performa
        \item \textbf{Model}: VGG19-LSTM vs BLIP
        \item \textbf{Dataset}: PathVQA dan VQA-RAD
    \end{itemize}

    \item "Bagaimana menerapkan model VQA terbaik untuk menjawab pertanyaan berdasarkan citra medis?"
    \begin{itemize}
        \item \textbf{Fokus}: Implementasi praktis
        \item \textbf{Output}: Sistem yang dapat digunakan
    \end{itemize}
\end{enumerate}

\addcontentsline{toc}{subsection}{Bagian 6 - Contoh Penelitian Lain: Studi Kasus Data Science}
\subsection*{Bagian 6 - Contoh Penelitian Lain: Studi Kasus Data Science}

Untuk memberikan perspektif yang lebih luas, mari kita analisis contoh penelitian lain di bidang data science.

\textbf{Contoh Judul Penelitian Data Science:}
"Prediksi Harga Saham menggunakan Long Short-Term Memory (LSTM) dengan Feature Engineering Berbasis Technical Indicators"

\textbf{Analisis Judul:}
\begin{itemize}
    \item \textbf{Task}: Prediksi (regression problem)
    \item \textbf{Domain}: Financial (harga saham)
    \item \textbf{Method}: LSTM (deep learning)
    \item \textbf{Innovation}: Feature engineering dengan technical indicators
    \item \textbf{Tipe}: Kuantitatif (karena fokus pada prediksi numerik)
\end{itemize}

\textbf{Struktur Latar Belakang untuk Contoh Ini:}

\textbf{Paragraf 1}: Konteks umum
\begin{lstlisting}[language=bash, style=bash, caption=Latar Belakang Penelitian Prediksi Saham]
"Pasar saham merupakan salah satu instrumen investasi yang paling
populer di dunia. Fluktuasi harga saham yang tidak dapat diprediksi
dengan mudah menjadi tantangan bagi investor dalam mengambil keputusan
investasi yang tepat."
\end{lstlisting}

\textbf{Paragraf 2}: Masalah spesifik
\begin{lstlisting}[language=bash, style=bash, caption=Identifikasi Masalah Spesifik]
"Metode prediksi tradisional seperti analisis fundamental dan teknikal
memiliki keterbatasan dalam menangkap pola kompleks yang terdapat dalam
data time series harga saham."
\end{lstlisting}

\textbf{Paragraf 3}: Solusi teknologi
\begin{lstlisting}[language=bash, style=bash, caption=Solusi yang Diusulkan]
"Deep learning, khususnya Long Short-Term Memory (LSTM), telah
menunjukkan kemampuan yang baik dalam memprediksi data time series
karena dapat menangkap dependensi jangka panjang dalam data."
\end{lstlisting}

\textbf{Rumusan Masalah untuk Contoh Ini:}
\begin{enumerate}
    \item Bagaimana merancang feature engineering yang optimal menggunakan technical indicators untuk prediksi harga saham?
    \item Bagaimana membangun model LSTM yang dapat memprediksi harga saham dengan akurasi tinggi?
    \item Bagaimana membandingkan performa model LSTM dengan metode prediksi tradisional?
\end{enumerate}

\textbf{Data Kuantitatif untuk Penelitian Ini:}
\begin{lstlisting}[language=python, style=python, caption=Contoh Data Kuantitatif Prediksi Saham]
# Data harga saham dan technical indicators
stock_data = {
    'date': ['2024-01-01', '2024-01-02', '2024-01-03'],
    'open_price': [4520.0, 4535.0, 4510.0],
    'close_price': [4530.0, 4525.0, 4540.0],
    'volume': [1250000, 1180000, 1320000],
    'rsi': [65.2, 62.8, 68.1],           # Relative Strength Index
    'macd': [12.5, 15.2, 8.9],          # Moving Average Convergence Divergence
    'bollinger_upper': [4580.0, 4575.0, 4590.0],
    'bollinger_lower': [4480.0, 4485.0, 4470.0]
}

# Hasil evaluasi model
model_evaluation = {
    'LSTM': {'MAE': 45.2, 'RMSE': 67.8, 'MAPE': 2.1},
    'ARIMA': {'MAE': 78.5, 'RMSE': 102.3, 'MAPE': 3.8},
    'Linear Regression': {'MAE': 89.1, 'RMSE': 125.6, 'MAPE': 4.2}
}
\end{lstlisting}

\addcontentsline{toc}{subsection}{Bagian 7 - Cara Menentukan Judul Penelitian yang Baik}
\subsection*{Bagian 7 - Cara Menentukan Judul Penelitian yang Baik}

\textbf{Kriteria Judul yang Baik:}

\begin{enumerate}
    \item \textbf{Spesifik dan Jelas}
    \begin{itemize}
        \item Hindari kata-kata umum seperti "Sistem", "Aplikasi" tanpa detail
        \item Sebutkan metode/teknologi yang digunakan
        \item Tentukan domain aplikasi yang spesifik
    \end{itemize}

    \item \textbf{Measurable (Dapat Diukur)}
    \begin{itemize}
        \item Gunakan kata seperti "Perbandingan", "Analisis", "Optimasi"
        \item Hindari kata subjektif seperti "Terbaik", "Optimal" tanpa kriteria
    \end{itemize}

    \item \textbf{Feasible (Dapat Dilaksanakan)}
    \begin{itemize}
        \item Sesuaikan dengan waktu dan resources yang tersedia
        \item Jangan terlalu ambisius untuk tingkat S1
    \end{itemize}

    \item \textbf{Novel (Ada Kebaruan)}
    \begin{itemize}
        \item Kombinasi metode yang belum pernah dicoba
        \item Aplikasi pada domain yang belum dieksplorasi
        \item Perbaikan pada metode yang sudah ada
    \end{itemize}
\end{enumerate}

\textbf{Template Formula Judul:}
\begin{lstlisting}[language=bash, style=bash, caption=Formula Judul Penelitian]
[TASK/PROBLEM] + [METHOD/TECHNIQUE] + [DOMAIN/APPLICATION] + [INNOVATION]

Contoh:
- Prediksi + LSTM + Harga Saham + Feature Engineering Technical Indicators
- Klasifikasi + CNN + Citra Medis + Transfer Learning
- Deteksi + YOLO + Objek + Real-time Processing
- Rekomendasi + Collaborative Filtering + E-commerce + Deep Learning
\end{lstlisting}

\textbf{Contoh Judul Berdasarkan Tipe Penelitian:}

\textbf{Kuantitatif:}
\begin{itemize}
    \item "Perbandingan Akurasi Algoritma Random Forest dan XGBoost untuk Prediksi Customer Churn"
    \item "Analisis Performa Algoritma Clustering pada Dataset Media Sosial Berukuran Besar"
    \item "Optimasi Hyperparameter Neural Network menggunakan Genetic Algorithm"
\end{itemize}

\textbf{Kualitatif:}
\begin{itemize}
    \item "Analisis User Experience pada Aplikasi Mobile Banking: Studi Fenomenologi"
    \item "Eksplorasi Faktor Adopsi Teknologi AI di Industri Manufaktur Indonesia"
    \item "Studi Etnografi Penggunaan Media Sosial untuk Pembelajaran Online"
\end{itemize}

\textbf{Mixed Methods:}
\begin{itemize}
    \item "Pengembangan Chatbot Customer Service menggunakan NLP: Analisis Performa dan User Satisfaction"
    \item "Sistem Rekomendasi Berbasis Machine Learning: Evaluasi Akurasi dan Studi User Experience"
\end{itemize}

\addcontentsline{toc}{subsection}{Bagian 8 - Metodologi Penelitian untuk Data Science}
\subsection*{Bagian 8 - Metodologi Penelitian untuk Data Science}

\textbf{Tahapan Penelitian Data Science:}

\textbf{1. Problem Definition}
\begin{itemize}
    \item \textbf{Business Understanding}: Memahami masalah dari perspektif bisnis/domain
    \item \textbf{Success Criteria}: Menentukan metrik keberhasilan
    \item \textbf{Constraints}: Mengidentifikasi keterbatasan (waktu, data, resources)
\end{itemize}

\textbf{2. Data Understanding}
\begin{itemize}
    \item \textbf{Data Collection}: Mengumpulkan data dari berbagai sumber
    \item \textbf{Data Exploration}: Exploratory Data Analysis (EDA)
    \item \textbf{Data Quality Assessment}: Mengecek kualitas dan kelengkapan data
\end{itemize}

\textbf{3. Data Preparation}
\begin{itemize}
    \item \textbf{Data Cleaning}: Menangani missing values, outliers, noise
    \item \textbf{Feature Engineering}: Membuat fitur baru dari data yang ada
    \item \textbf{Data Transformation}: Normalisasi, encoding, scaling
\end{itemize}

\textbf{4. Modeling}
\begin{itemize}
    \item \textbf{Algorithm Selection}: Memilih algoritma yang sesuai
    \item \textbf{Model Training}: Melatih model dengan data training
    \item \textbf{Hyperparameter Tuning}: Optimasi parameter model
\end{itemize}

\textbf{5. Evaluation}
\begin{itemize}
    \item \textbf{Performance Metrics}: Mengukur performa dengan metrik yang tepat
    \item \textbf{Cross Validation}: Validasi model dengan teknik yang robust
    \item \textbf{Model Comparison}: Membandingkan dengan baseline atau model lain
\end{itemize}

\textbf{6. Deployment}
\begin{itemize}
    \item \textbf{System Integration}: Mengintegrasikan model ke dalam sistem
    \item \textbf{User Interface}: Membuat interface untuk interaksi pengguna
    \item \textbf{Performance Monitoring}: Monitoring performa sistem di production
\end{itemize}

\textbf{Workflow Penelitian Data Science:}
\begin{lstlisting}[language=python, style=python, caption=Workflow Penelitian Data Science]
# 1. Data Collection
import pandas as pd
import numpy as np
from sklearn.model_selection import train_test_split
from sklearn.preprocessing import StandardScaler

# Load dataset
data = pd.read_csv('dataset.csv')
print(f"Dataset shape: {data.shape}")

# 2. Exploratory Data Analysis
print("Dataset Info:")
print(data.info())
print("\nStatistical Summary:")
print(data.describe())

# 3. Data Preprocessing
# Handle missing values
data = data.dropna()

# Feature engineering
data['new_feature'] = data['feature1'] * data['feature2']

# Split data
X = data.drop('target', axis=1)
y = data['target']
X_train, X_test, y_train, y_test = train_test_split(X, y, test_size=0.2, random_state=42)

# 4. Model Training
from sklearn.ensemble import RandomForestClassifier
from sklearn.metrics import accuracy_score, classification_report

model = RandomForestClassifier(n_estimators=100, random_state=42)
model.fit(X_train, y_train)

# 5. Model Evaluation
y_pred = model.predict(X_test)
accuracy = accuracy_score(y_test, y_pred)
print(f"Accuracy: {accuracy:.4f}")
print("\nClassification Report:")
print(classification_report(y_test, y_pred))
\end{lstlisting}

\addcontentsline{toc}{subsection}{Bagian 9 - Tabel-Tabel Penting dalam BAB III}
\subsection*{Bagian 9 - Tabel-Tabel Penting dalam BAB III}

Berdasarkan template, berikut adalah tabel-tabel yang biasanya ada di BAB III:

\textbf{Tabel 3.1: Jadwal Pelaksanaan Penelitian}
\begin{itemize}
    \item \textbf{Format}: Gantt chart dengan bulan dan aktivitas
    \item \textbf{Isi}: Timeline setiap tahap penelitian
    \item \textbf{Fungsi}: Menunjukkan perencanaan waktu yang sistematis
\end{itemize}

\textbf{Tabel 3.2-3.5: Proses Preprocessing Text}
\begin{itemize}
    \item \textbf{Tabel 3.2}: "Proses case folding"
    \item \textbf{Tabel 3.3}: "Proses remove punctuation"
    \item \textbf{Tabel 3.4}: "Proses stemming"
    \item \textbf{Tabel 3.5}: "Proses tokenisasi"
\end{itemize}

\textbf{Format Tabel Preprocessing:}
\begin{lstlisting}[language=bash, style=bash, caption=Format Tabel Preprocessing]
\begin{table}[H]
    \centering
    \caption{Proses case folding}
    \label{tab:case-folding}
    \begin{tabular}{|l|l|}
    \hline
    \multicolumn{1}{|c|}{\textbf{Sebelum}} & \multicolumn{1}{c|}{\textbf{Sesudah}} \\ \hline
    The Quick Brown Fox JUMPS              & the quick brown fox jumps             \\ \hline
    \end{tabular}
\end{table}
\end{lstlisting}

\textbf{Tabel Tambahan yang Sering Digunakan:}
\begin{itemize}
    \item \textbf{Spesifikasi Hardware}: Detail komputer/server yang digunakan
    \item \textbf{Spesifikasi Software}: Versi OS, framework, library
    \item \textbf{Karakteristik Dataset}: Jumlah data, distribusi kelas, ukuran file
    \item \textbf{Konfigurasi Eksperimen}: Parameter model, hyperparameter
\end{itemize}

\addcontentsline{toc}{subsection}{Bagian 10 - Workflow Penelitian: Dari Ide hingga Implementasi}
\subsection*{Bagian 10 - Workflow Penelitian: Dari Ide hingga Implementasi}

\textbf{Pertanyaan Penting: Mulai dari Mana?}

Untuk penelitian di bidang data science, ada beberapa pendekatan:

\textbf{Pendekatan 1: Problem-First (Direkomendasikan)}
\begin{enumerate}
    \item \textbf{Identifikasi Masalah}: Cari masalah nyata yang perlu diselesaikan
    \item \textbf{Literature Review}: Pelajari solusi yang sudah ada
    \item \textbf{Gap Analysis}: Temukan kekurangan dari solusi existing
    \item \textbf{Data Search}: Cari dataset yang sesuai dengan masalah
    \item \textbf{Method Selection}: Pilih metode yang tepat
    \item \textbf{Implementation}: Implementasi dan evaluasi
\end{enumerate}

\textbf{Pendekatan 2: Data-First}
\begin{enumerate}
    \item \textbf{Dataset Discovery}: Temukan dataset menarik yang tersedia
    \item \textbf{Data Exploration}: Eksplorasi karakteristik dataset
    \item \textbf{Problem Formulation}: Rumuskan masalah berdasarkan data
    \item \textbf{Method Development}: Kembangkan metode untuk menyelesaikan masalah
    \item \textbf{Evaluation}: Evaluasi dan analisis hasil
\end{enumerate}

\textbf{Pendekatan 3: Method-First}
\begin{enumerate}
    \item \textbf{Method Study}: Pelajari metode/algoritma baru yang menarik
    \item \textbf{Application Domain}: Cari domain aplikasi yang sesuai
    \item \textbf{Dataset Search}: Cari dataset untuk testing metode
    \item \textbf{Implementation}: Implementasi metode
    \item \textbf{Comparison}: Bandingkan dengan metode existing
\end{enumerate}

\textbf{Rekomendasi untuk Mahasiswa:}
\begin{lstlisting}[language=bash, style=bash, caption=Workflow yang Direkomendasikan]
TAHAP 1: PERSIAPAN (Bulan 1-2)
1. Identifikasi minat dan passion Anda
2. Diskusi dengan dosen pembimbing
3. Literature review awal (baca 20-30 paper)
4. Tentukan domain penelitian (CV, NLP, Data Mining, dll)

TAHAP 2: PERENCANAAN (Bulan 3)
1. Rumuskan masalah spesifik
2. Cari dan evaluasi dataset yang tersedia
3. Tentukan metode yang akan digunakan
4. Buat proposal penelitian

TAHAP 3: IMPLEMENTASI (Bulan 4-6)
1. Data collection dan preprocessing
2. Implementasi baseline model
3. Eksperimen dengan berbagai konfigurasi
4. Evaluasi dan analisis hasil

TAHAP 4: PENULISAN (Bulan 7-8)
1. Tulis draft BAB I-III
2. Tulis BAB IV berdasarkan hasil eksperimen
3. Tulis BAB V dan abstrak
4. Review dan revisi keseluruhan
\end{lstlisting}

\addcontentsline{toc}{subsection}{Bagian 11 - Data Science: Kualitatif vs Kuantitatif}
\subsection*{Bagian 11 - Data Science: Kualitatif vs Kuantitatif}

\textbf{Data Science Kuantitatif}

\textbf{Karakteristik:}
\begin{itemize}
    \item \textbf{Fokus}: Prediksi, klasifikasi, clustering, optimasi
    \item \textbf{Data}: Numerical, structured data
    \item \textbf{Metrik}: Accuracy, precision, recall, F1-score, RMSE, MAE
    \item \textbf{Tools}: Python/R, scikit-learn, TensorFlow, PyTorch
\end{itemize}

\textbf{Contoh Penelitian:}
\begin{itemize}
    \item Prediksi harga rumah menggunakan regression
    \item Klasifikasi email spam menggunakan NLP
    \item Clustering customer untuk segmentasi pasar
    \item Deteksi fraud dalam transaksi keuangan
\end{itemize}

\textbf{Contoh Dataset Kuantitatif:}
\begin{lstlisting}[language=python, style=python, caption=Dataset Kuantitatif untuk Data Science]
# Dataset untuk prediksi harga rumah
house_data = {
    'bedrooms': [3, 4, 2, 5, 3],
    'bathrooms': [2, 3, 1, 4, 2],
    'sqft_living': [1800, 2400, 1200, 3200, 1900],
    'sqft_lot': [7200, 9600, 4800, 12800, 8100],
    'floors': [1, 2, 1, 2, 1],
    'waterfront': [0, 1, 0, 1, 0],
    'condition': [3, 4, 3, 5, 4],
    'grade': [7, 8, 6, 9, 7],
    'price': [450000, 680000, 320000, 950000, 520000]  # Target variable
}

# Dataset untuk klasifikasi sentimen
sentiment_data = {
    'text': [
        "Produk ini sangat bagus dan berkualitas",
        "Pelayanan mengecewakan, tidak recommended",
        "Harga sesuai dengan kualitas yang diberikan"
    ],
    'sentiment': ['positive', 'negative', 'neutral'],
    'confidence': [0.95, 0.87, 0.72]
}
\end{lstlisting}

\textbf{Data Science Kualitatif}

\textbf{Karakteristik:}
\begin{itemize}
    \item \textbf{Fokus}: Understanding, exploration, interpretation
    \item \textbf{Data}: Text, images, audio, behavioral data
    \item \textbf{Metrik}: Thematic analysis, content analysis, user satisfaction
    \item \textbf{Tools}: NVivo, Atlas.ti, Python (untuk text analysis)
\end{itemize}

\textbf{Contoh Penelitian:}
\begin{itemize}
    \item Analisis sentimen media sosial tentang kebijakan pemerintah
    \item Studi user behavior pada aplikasi e-commerce
    \item Analisis konten untuk memahami trend topik
    \item Evaluasi usability sistem informasi
\end{itemize}

\textbf{Contoh Dataset Kualitatif:}
\begin{lstlisting}[language=python, style=python, caption=Dataset Kualitatif untuk Data Science]
# Dataset untuk analisis sentimen media sosial
social_media_data = [
    {
        'post_id': 'P001',
        'platform': 'Twitter',
        'text': '''Kebijakan baru pemerintah tentang pajak digital sangat
                  memberatkan UMKM. Harusnya ada insentif untuk bisnis kecil.''',
        'timestamp': '2024-01-15 10:30:00',
        'likes': 245,
        'retweets': 67,
        'comments': 89,
        'user_type': 'Business Owner'
    },
    {
        'post_id': 'P002',
        'platform': 'Facebook',
        'text': '''Senang dengan kemudahan pembayaran digital sekarang.
                  Transaksi jadi lebih cepat dan aman.''',
        'timestamp': '2024-01-15 14:20:00',
        'likes': 156,
        'shares': 23,
        'comments': 34,
        'user_type': 'General Public'
    }
]

# Dataset untuk user experience study
ux_study_data = {
    'session_id': 'S001',
    'user_profile': {
        'age': 25,
        'gender': 'Female',
        'tech_savviness': 'Medium',
        'previous_experience': 'First time user'
    },
    'task_performance': {
        'task_completion_time': 180,  # seconds
        'errors_made': 3,
        'help_requests': 2,
        'success_rate': 0.8
    },
    'qualitative_feedback': '''
        Interface cukup intuitif tapi ada beberapa button yang membingungkan.
        Loading time agak lama untuk upload file. Overall experience cukup baik
        tapi perlu improvement di beberapa area.
    ''',
    'satisfaction_score': 7,  # 1-10 scale
    'recommendations': [
        'Improve button labeling',
        'Add progress indicators',
        'Provide better error messages'
    ]
}
\end{lstlisting}

\addcontentsline{toc}{subsection}{Bagian 12 - Best Practices Penulisan TA}
\subsection*{Bagian 12 - Best Practices Penulisan TA}

\textbf{1. Penggunaan Gambar}

\textbf{Kapan Menggunakan Gambar:}
\begin{itemize}
    \item \textbf{Arsitektur Model}: Diagram neural network, system architecture
    \item \textbf{Workflow}: Flowchart metodologi, process diagram
    \item \textbf{Hasil Visualisasi}: Plot performa, confusion matrix, ROC curve
    \item \textbf{Interface}: Screenshot aplikasi, mockup design
    \item \textbf{Konsep}: Diagram konsep teoritis, mathematical models
\end{itemize}

\textbf{Format Gambar yang Baik:}
\begin{lstlisting}[language=bash, style=bash, caption=Best Practice Gambar dalam LaTeX]
\begin{figure}[H]
    \centering
    \includegraphics[width=0.8\textwidth]{image/path/filename.png}
    \caption{Arsitektur model CNN untuk klasifikasi citra medis}
    \label{fig:cnn-architecture}
\end{figure}

% Referensi gambar dalam teks:
Arsitektur CNN yang digunakan dalam penelitian ini dapat dilihat
pada Gambar \ref{fig:cnn-architecture}.
\end{lstlisting}

\textbf{2. Penggunaan Tabel}

\textbf{Kapan Menggunakan Tabel:}
\begin{itemize}
    \item \textbf{Data Eksperimen}: Hasil perbandingan model, metrik evaluasi
    \item \textbf{Spesifikasi}: Hardware, software, dataset characteristics
    \item \textbf{Preprocessing}: Contoh transformasi data step-by-step
    \item \textbf{Hyperparameter}: Konfigurasi model yang diuji
    \item \textbf{Comparison}: Perbandingan dengan penelitian lain
\end{itemize}

\textbf{3. Penggunaan Kode}

\textbf{Prinsip Penggunaan Kode:}
\begin{enumerate}
    \item \textbf{Selective}: Hanya kode yang penting dan berkontribusi
    \item \textbf{Clean}: Kode harus rapi, terkomenter, dan mudah dipahami
    \item \textbf{Complete}: Jangan potong kode di tengah-tengah fungsi
    \item \textbf{Relevant}: Kode harus relevan dengan pembahasan
\end{enumerate}

\textbf{Jenis Kode yang Perlu Ditampilkan:}
\begin{itemize}
    \item \textbf{Model Architecture}: Definisi model neural network
    \item \textbf{Key Algorithms}: Algoritma utama yang dikembangkan
    \item \textbf{Preprocessing Functions}: Fungsi untuk data cleaning
    \item \textbf{Evaluation Code}: Kode untuk menghitung metrik
    \item \textbf{API Endpoints}: Interface sistem yang dikembangkan
\end{itemize}

\textbf{Yang TIDAK Perlu Ditampilkan:}
\begin{itemize}
    \item Import statements yang panjang
    \item Kode debugging atau testing
    \item Kode yang sudah standard (seperti train\_test\_split)
    \item Kode untuk plotting yang sederhana
\end{itemize}

\addcontentsline{toc}{subsection}{Bagian 13 - Sitasi dan Referensi}
\subsection*{Bagian 13 - Sitasi dan Referensi}

\textbf{Sistem Sitasi yang Digunakan:}
Berdasarkan template, menggunakan sistem author-year dengan natbib.

\textbf{Format Sitasi dalam Teks:}
\begin{lstlisting}[language=bash, style=bash, caption=Contoh Sitasi dalam Teks]
% Sitasi tunggal
Citra medis menjadi fokus penelitian utama \citep{kusuma2020penerapan}.

% Sitasi multiple
Beberapa penelitian menunjukkan efektivitas deep learning
\citep{alom2019state, wardani2023klasifikasi, kristiyanti2023machine}.

% Sitasi dengan halaman
Menurut \citet{goodfellow2016deep}, deep learning memiliki kemampuan...

% Sitasi dalam kurung
Deep learning telah terbukti efektif dalam berbagai aplikasi
(lihat \citealp{lecun2015deep} untuk review komprehensif).
\end{lstlisting}

\textbf{Jenis Referensi yang Dibutuhkan:}
\begin{enumerate}
    \item \textbf{Journal Articles}: Paper dari jurnal internasional bereputasi
    \item \textbf{Conference Papers}: Paper dari konferensi internasional
    \item \textbf{Books}: Textbook atau monograph yang relevan
    \item \textbf{Thesis/Dissertation}: Penelitian serupa dari universitas lain
    \item \textbf{Technical Reports}: Laporan teknis dari institusi penelitian
    \item \textbf{Online Resources}: Dataset documentation, API documentation
\end{enumerate}

\textbf{Kualitas Referensi:}
\begin{itemize}
    \item \textbf{Minimal 50 referensi} untuk TA S1
    \item \textbf{80\% dari jurnal/konferensi} internasional
    \item \textbf{Terbaru}: Minimal 70\% dari 5 tahun terakhir
    \item \textbf{Relevan}: Semua referensi harus relevan dengan topik
\end{itemize}

\addcontentsline{toc}{subsection}{Bagian 14 - Contoh Lengkap: Analisis Template VQA}
\subsection*{Bagian 14 - Contoh Lengkap: Analisis Template VQA}

Mari kita analisis secara detail template TA yang Anda berikan:

\textbf{Kekuatan Template VQA:}

\textbf{1. Judul yang Excellent}
\begin{itemize}
    \item \textbf{Comparison Study}: Jelas ini penelitian perbandingan
    \item \textbf{Specific Methods}: VGG19-LSTM vs BLIP (tidak ambigu)
    \item \textbf{Clear Task}: Visual Question Answering
    \item \textbf{Domain}: Medical images (spesifik)
\end{itemize}

\textbf{2. Latar Belakang yang Terstruktur}
\begin{itemize}
    \item \textbf{Funnel Approach}: Dari umum (citra medis) ke spesifik (VQA)
    \item \textbf{Problem Justification}: Jelas mengapa masalah ini penting
    \item \textbf{Solution Motivation}: Logis mengapa VQA adalah solusi
    \item \textbf{Research Positioning}: Jelas kontribusi penelitian
\end{itemize}

\textbf{3. Metodologi yang Komprehensif}
\begin{itemize}
    \item \textbf{Clear Timeline}: Jadwal penelitian yang realistis
    \item \textbf{Detailed Specs}: Spesifikasi hardware dan software lengkap
    \item \textbf{Step-by-step}: Metodologi dijelaskan tahap demi tahap
    \item \textbf{Visual Workflow}: Diagram alir yang membantu pemahaman
\end{itemize}

\textbf{4. Hasil yang Comprehensive}
\begin{itemize}
    \item \textbf{Multiple Metrics}: Akurasi, BLEU score, inference time
    \item \textbf{Detailed Comparison}: Perbandingan dengan berbagai konfigurasi
    \item \textbf{Practical Implementation}: Sistem web yang dapat digunakan
    \item \textbf{Performance Analysis}: Analisis mendalam tentang hasil
\end{itemize}

\textbf{Pembelajaran dari Template:}

\textbf{1. Struktur Eksperimen yang Baik}
\begin{lstlisting}[language=python, style=python, caption=Struktur Eksperimen dari Template]
# Konfigurasi eksperimen yang sistematis
experiment_config = {
    'datasets': ['PathVQA', 'VQA-RAD'],
    'models': ['VGG19-LSTM', 'BLIP'],
    'epochs': [15, 30, 45],
    'batch_sizes': [8, 16, 32],
    'learning_rates': [1e-5, 5e-5],
    'metrics': ['accuracy', 'bleu_1', 'bleu_2', 'bleu_3']
}

# Total eksperimen: 2 datasets × 2 models × 3 epochs × 3 batch_sizes × 2 lr
# = 72 eksperimen (sangat komprehensif!)
\end{lstlisting}

\textbf{2. Preprocessing yang Sistematis}
Template menunjukkan setiap tahap preprocessing dengan tabel:
\begin{itemize}
    \item \textbf{Case folding}: "The Quick Brown Fox" → "the quick brown fox"
    \item \textbf{Remove punctuation}: "Hello!" → "Hello"
    \item \textbf{Stemming}: "running" → "run"
    \item \textbf{Tokenization}: "hello world" → ["hello", "world"]
\end{itemize}

\textbf{3. Evaluasi yang Robust}
\begin{itemize}
    \item \textbf{Multiple Metrics}: Tidak hanya akurasi, tapi juga BLEU score
    \item \textbf{Different Question Types}: Close-ended vs open-ended
    \item \textbf{Model Optimization}: Quantization untuk efisiensi
    \item \textbf{Practical Deployment}: Implementasi dalam sistem web
\end{itemize}

\addcontentsline{toc}{subsection}{Bagian 15 - Contoh Penelitian Lain: Sistem Rekomendasi}
\subsection*{Bagian 15 - Contoh Penelitian Lain: Sistem Rekomendasi}

Untuk memberikan perspektif yang lebih luas, mari kita buat contoh penelitian lain:

\textbf{Judul Contoh:}
"Pengembangan Sistem Rekomendasi Film menggunakan Hybrid Collaborative Filtering dan Content-Based Filtering: Analisis Performa dan User Satisfaction"

\textbf{Analisis Judul:}
\begin{itemize}
    \item \textbf{Task}: Pengembangan sistem rekomendasi
    \item \textbf{Method}: Hybrid approach (CF + CBF)
    \item \textbf{Domain}: Film/movie recommendation
    \item \textbf{Evaluation}: Performa (kuantitatif) + User satisfaction (kualitatif)
    \item \textbf{Tipe}: Mixed methods research
\end{itemize}

\textbf{Struktur Latar Belakang untuk Contoh Ini:}

\textbf{Paragraf 1}: Konteks industri
\begin{lstlisting}[language=bash, style=bash, caption=Latar Belakang Sistem Rekomendasi]
"Industri streaming film mengalami pertumbuhan pesat dengan jutaan
konten yang tersedia. Platform seperti Netflix, Disney+, dan Amazon
Prime memiliki katalog yang sangat besar sehingga pengguna kesulitan
menemukan konten yang sesuai dengan preferensi mereka."
\end{lstlisting}

\textbf{Paragraf 2}: Masalah information overload
\begin{lstlisting}[language=bash, style=bash, caption=Masalah Information Overload]
"Information overload menjadi masalah utama dimana pengguna menghabiskan
lebih banyak waktu untuk mencari film daripada menontonnya. Penelitian
menunjukkan bahwa 60% pengguna meninggalkan platform karena kesulitan
menemukan konten yang menarik."
\end{lstlisting}

\textbf{Paragraf 3}: Keterbatasan solusi existing
\begin{lstlisting}[language=bash, style=bash, caption=Keterbatasan Solusi Existing]
"Sistem rekomendasi tradisional yang hanya menggunakan collaborative
filtering memiliki masalah cold start problem dan sparsity. Sementara
content-based filtering terbatas pada similarity yang dapat diukur
dari metadata film."
\end{lstlisting}

\textbf{Paragraf 4}: Solusi hybrid
\begin{lstlisting}[language=bash, style=bash, caption=Solusi Hybrid yang Diusulkan]
"Pendekatan hybrid yang menggabungkan collaborative filtering dan
content-based filtering dapat mengatasi keterbatasan masing-masing
metode dan memberikan rekomendasi yang lebih akurat dan diverse."
\end{lstlisting}

\textbf{Data untuk Penelitian Sistem Rekomendasi:}

\textbf{Data Kuantitatif:}
\begin{lstlisting}[language=python, style=python, caption=Data Kuantitatif Sistem Rekomendasi]
# Dataset rating film
ratings_data = {
    'user_id': [1, 1, 1, 2, 2, 3, 3, 3],
    'movie_id': [101, 102, 103, 101, 104, 102, 103, 105],
    'rating': [4.5, 3.0, 5.0, 4.0, 2.5, 3.5, 4.5, 5.0],
    'timestamp': ['2024-01-01', '2024-01-02', '2024-01-03',
                  '2024-01-01', '2024-01-02', '2024-01-01',
                  '2024-01-02', '2024-01-03']
}

# Metrik evaluasi sistem rekomendasi
evaluation_metrics = {
    'Collaborative Filtering': {
        'RMSE': 0.85,
        'MAE': 0.67,
        'Precision@10': 0.72,
        'Recall@10': 0.68,
        'F1@10': 0.70,
        'Coverage': 0.45
    },
    'Content-Based': {
        'RMSE': 0.92,
        'MAE': 0.74,
        'Precision@10': 0.68,
        'Recall@10': 0.71,
        'F1@10': 0.69,
        'Coverage': 0.89
    },
    'Hybrid Method': {
        'RMSE': 0.78,
        'MAE': 0.61,
        'Precision@10': 0.76,
        'Recall@10': 0.73,
        'F1@10': 0.74,
        'Coverage': 0.67
    }
}
\end{lstlisting}

\textbf{Data Kualitatif:}
\begin{lstlisting}[language=python, style=python, caption=Data Kualitatif User Satisfaction]
# Hasil survey user satisfaction
user_satisfaction_survey = [
    {
        'user_id': 'U001',
        'age_group': '18-25',
        'usage_frequency': 'Daily',
        'satisfaction_score': 8,
        'feedback': '''Rekomendasi sangat akurat dan sesuai selera.
                      Suka dengan variasi genre yang diberikan.
                      Kadang ada film lama yang tidak expected tapi
                      ternyata bagus.''',
        'liked_features': ['Accurate recommendations', 'Genre diversity'],
        'disliked_features': ['Sometimes too many old movies'],
        'improvement_suggestions': ['Add mood-based recommendations']
    },
    {
        'user_id': 'U002',
        'age_group': '26-35',
        'usage_frequency': 'Weekly',
        'satisfaction_score': 7,
        'feedback': '''Sistem rekomendasi cukup baik tapi kadang
                      merekomendasikan film yang sudah pernah ditonton.
                      Perlu ada filter untuk exclude watched movies.''',
        'liked_features': ['Good accuracy', 'Fast response'],
        'disliked_features': ['Recommends watched movies', 'Limited explanation'],
        'improvement_suggestions': ['Better filtering', 'Explain why recommended']
    }
]
\end{lstlisting}

\addcontentsline{toc}{subsection}{Bagian 16 - Checklist Kelengkapan TA}
\subsection*{Bagian 16 - Checklist Kelengkapan TA}

\textbf{Checklist Bagian Awal:}
\begin{enumerate}
    \item[$\square$] Halaman judul dengan format yang benar
    \item[$\square$] Halaman pengesahan (akan diisi saat sidang)
    \item[$\square$] Pernyataan bebas plagiasi (ditandatangani + materai)
    \item[$\square$] Surat pernyataan publikasi
    \item[$\square$] Abstrak Indonesia (max 250 kata + kata kunci)
    \item[$\square$] Abstract English (konsisten dengan abstrak Indonesia)
    \item[$\square$] Kata pengantar (ucapan terima kasih)
    \item[$\square$] Daftar isi (otomatis dari LaTeX)
    \item[$\square$] Daftar tabel (otomatis dari LaTeX)
    \item[$\square$] Daftar gambar (otomatis dari LaTeX)
    \item[$\square$] Daftar lampiran (otomatis dari LaTeX)
    \item[$\square$] Daftar singkatan (alfabetis)
\end{enumerate}

\textbf{Checklist BAB I:}
\begin{enumerate}
    \item[$\square$] Latar belakang (2-3 halaman, min 10 sitasi)
    \item[$\square$] Rumusan masalah (3-5 pertanyaan spesifik)
    \item[$\square$] Tujuan penelitian (sesuai dengan rumusan masalah)
    \item[$\square$] Manfaat penelitian (teoritis + praktis)
\end{enumerate}

\textbf{Checklist BAB II:}
\begin{enumerate}
    \item[$\square$] Konsep dasar domain penelitian
    \item[$\square$] Teori/metode yang digunakan (dengan gambar arsitektur)
    \item[$\square$] Metrik evaluasi (dengan rumus matematika)
    \item[$\square$] Dataset yang digunakan
    \item[$\square$] Penelitian terkait (positioning penelitian)
    \item[$\square$] Minimal 30 referensi
\end{enumerate}

\textbf{Checklist BAB III:}
\begin{enumerate}
    \item[$\square$] Waktu dan lokasi penelitian
    \item[$\square$] Jadwal pelaksanaan (tabel Gantt)
    \item[$\square$] Spesifikasi hardware dan software
    \item[$\square$] Diagram alir metodologi penelitian
    \item[$\square$] Penjelasan setiap tahap metodologi
    \item[$\square$] Tabel preprocessing (jika ada)
\end{enumerate}

\textbf{Checklist BAB IV:}
\begin{enumerate}
    \item[$\square$] Deskripsi dataset dengan statistik
    \item[$\square$] Implementasi model (dengan kode penting)
    \item[$\square$] Hasil eksperimen (tabel dan grafik)
    \item[$\square$] Analisis dan pembahasan hasil
    \item[$\square$] Implementasi sistem (jika ada)
    \item[$\square$] Screenshot interface (jika ada)
\end{enumerate}

\textbf{Checklist BAB V:}
\begin{enumerate}
    \item[$\square$] Kesimpulan (jawab semua rumusan masalah)
    \item[$\square$] Saran untuk penelitian selanjutnya
    \item[$\square$] Keterbatasan penelitian
\end{enumerate}

\textbf{Checklist Bagian Akhir:}
\begin{enumerate}
    \item[$\square$] Daftar pustaka (min 50 referensi, format konsisten)
    \item[$\square$] Lampiran (kode lengkap, data tambahan, biodata)
\end{enumerate}

\addcontentsline{toc}{subsection}{Bagian 17 - Tips Praktis Penulisan}
\subsection*{Bagian 17 - Tips Praktis Penulisan}

\textbf{Tips Menulis yang Efektif:}

\textbf{1. Bahasa dan Gaya Penulisan}
\begin{itemize}
    \item \textbf{Formal}: Gunakan bahasa formal dan objektif
    \item \textbf{Konsisten}: Gunakan istilah yang konsisten sepanjang dokumen
    \item \textbf{Aktif}: Gunakan kalimat aktif daripada pasif
    \item \textbf{Jelas}: Hindari kalimat yang ambigu atau bertele-tele
\end{itemize}

\textbf{2. Organisasi Konten}
\begin{itemize}
    \item \textbf{Logical Flow}: Setiap paragraf harus terhubung logis
    \item \textbf{Transition}: Gunakan kata penghubung antar paragraf
    \item \textbf{Evidence-Based}: Setiap klaim harus didukung referensi
    \item \textbf{Balanced}: Jangan terlalu fokus pada satu aspek
\end{itemize}

\textbf{3. Manajemen Referensi}
\begin{lstlisting}[language=bash, style=bash, caption=Tips Manajemen Referensi]
% Gunakan tools seperti Mendeley atau Zotero
% Export ke format BibTeX untuk LaTeX

% Contoh entry BibTeX yang baik:
@article{smith2023deep,
  title={Deep Learning for Medical Image Analysis: A Comprehensive Review},
  author={Smith, John and Johnson, Mary},
  journal={IEEE Transactions on Medical Imaging},
  volume={42},
  number={3},
  pages={123--145},
  year={2023},
  publisher={IEEE}
}
\end{lstlisting}

\textbf{4. Version Control}
\begin{itemize}
    \item \textbf{Git}: Gunakan Git untuk version control
    \item \textbf{Backup}: Backup reguler ke cloud storage
    \item \textbf{Naming}: Gunakan naming convention yang konsisten untuk file
\end{itemize}

\addcontentsline{toc}{subsection}{Bagian 18 - Common Mistakes dan Cara Menghindarinya}
\subsection*{Bagian 18 - Common Mistakes dan Cara Menghindarinya}

\textbf{Kesalahan Umum dalam TA:}

\textbf{1. Judul yang Terlalu Umum}
\begin{itemize}
    \item \textbf{Salah}: "Sistem Informasi Akademik Berbasis Web"
    \item \textbf{Benar}: "Pengembangan Sistem Informasi Akademik dengan Fitur Prediksi Kelulusan menggunakan Machine Learning"
\end{itemize}

\textbf{2. Rumusan Masalah yang Tidak Spesifik}
\begin{itemize}
    \item \textbf{Salah}: "Bagaimana membuat sistem yang baik?"
    \item \textbf{Benar}: "Bagaimana mengoptimalkan akurasi model CNN untuk klasifikasi citra dengan dataset yang terbatas?"
\end{itemize}

\textbf{3. Metodologi yang Tidak Detail}
\begin{itemize}
    \item \textbf{Salah}: "Menggunakan machine learning untuk prediksi"
    \item \textbf{Benar}: "Menggunakan Random Forest dengan 100 trees, max\_depth=10, dan 5-fold cross validation"
\end{itemize}

\textbf{4. Evaluasi yang Tidak Komprehensif}
\begin{itemize}
    \item \textbf{Salah}: Hanya menggunakan akurasi
    \item \textbf{Benar}: Menggunakan multiple metrics (precision, recall, F1-score, confusion matrix)
\end{itemize}

\textbf{5. Sitasi yang Tidak Proper}
\begin{itemize}
    \item \textbf{Salah}: Menggunakan Wikipedia atau blog sebagai referensi utama
    \item \textbf{Benar}: Menggunakan jurnal peer-reviewed dan konferensi internasional
\end{itemize}

\addcontentsline{toc}{subsection}{Bagian 19 - Timeline Realistis untuk TA}
\subsection*{Bagian 19 - Timeline Realistis untuk TA}

\textbf{Timeline 8 Bulan (Semester 7-8):}

\textbf{Bulan 1-2: Persiapan dan Perencanaan}
\begin{itemize}
    \item Identifikasi minat dan passion
    \item Literature review awal (20-30 paper)
    \item Diskusi dengan calon pembimbing
    \item Penentuan topik dan scope penelitian
    \item Pencarian dataset awal
\end{itemize}

\textbf{Bulan 3: Proposal dan Persiapan}
\begin{itemize}
    \item Penulisan proposal penelitian
    \item Finalisasi dataset dan tools
    \item Setup environment development
    \item Penulisan BAB I dan II (draft awal)
\end{itemize}

\textbf{Bulan 4-5: Implementasi dan Eksperimen}
\begin{itemize}
    \item Data preprocessing dan exploration
    \item Implementasi baseline model
    \item Eksperimen dengan berbagai konfigurasi
    \item Dokumentasi hasil eksperimen
\end{itemize}

\textbf{Bulan 6: Analisis dan Pengembangan}
\begin{itemize}
    \item Analisis mendalam hasil eksperimen
    \item Pengembangan sistem/aplikasi (jika ada)
    \item Evaluasi komprehensif
    \item Persiapan untuk penulisan
\end{itemize}

\textbf{Bulan 7: Penulisan Intensif}
\begin{itemize}
    \item Penulisan BAB IV (hasil dan pembahasan)
    \item Revisi BAB I-III berdasarkan hasil
    \item Penulisan BAB V (kesimpulan dan saran)
    \item Penulisan abstrak dan kata pengantar
\end{itemize}

\textbf{Bulan 8: Finalisasi dan Sidang}
\begin{itemize}
    \item Review dan revisi keseluruhan
    \item Persiapan presentasi sidang
    \item Sidang TA
    \item Revisi berdasarkan masukan penguji
\end{itemize}

\addcontentsline{toc}{subsection}{Bagian 20 - Kesimpulan dan Action Plan}
\subsection*{Bagian 20 - Kesimpulan dan Action Plan}

\textbf{Ringkasan Pembelajaran:}

Setelah mempelajari modul ini, Anda seharusnya memahami:
\begin{enumerate}
    \item \textbf{Konsep TA}: Pengertian, tujuan, dan karakteristik TA Informatika
    \item \textbf{Jenis Penelitian}: Perbedaan kualitatif, kuantitatif, dan mixed methods
    \item \textbf{Struktur Penulisan}: Setiap bagian TA dan fungsinya
    \item \textbf{Metodologi}: Workflow penelitian data science
    \item \textbf{Best Practices}: Penggunaan gambar, tabel, kode, dan sitasi
    \item \textbf{Quality Standards}: Kriteria TA yang baik dan berkualitas
\end{enumerate}

\textbf{Action Plan untuk Memulai TA:}

\textbf{Minggu 1-2: Eksplorasi}
\begin{enumerate}
    \item Baca 10 paper terbaru di bidang yang diminati
    \item Identifikasi 3-5 topik potensial
    \item Diskusi dengan senior atau dosen
    \item Evaluasi ketersediaan dataset
\end{enumerate}

\textbf{Minggu 3-4: Fokus}
\begin{enumerate}
    \item Pilih 1 topik yang paling feasible
    \item Buat outline penelitian
    \item Cari pembimbing yang sesuai
    \item Mulai penulisan proposal
\end{enumerate}

\textbf{Minggu 5-8: Implementasi}
\begin{enumerate}
    \item Setup environment dan tools
    \item Mulai eksperimen awal
    \item Dokumentasi setiap langkah
    \item Regular meeting dengan pembimbing
\end{enumerate}

\textbf{Pesan Penutup:}

Tugas Akhir adalah journey yang menantang tapi sangat bermanfaat. Kunci sukses adalah:
\begin{itemize}
    \item \textbf{Konsistensi}: Kerja sedikit tapi rutin setiap hari
    \item \textbf{Dokumentasi}: Catat setiap eksperimen dan hasil
    \item \textbf{Komunikasi}: Regular diskusi dengan pembimbing
    \item \textbf{Fleksibilitas}: Siap mengubah arah jika diperlukan
    \item \textbf{Persistence}: Jangan menyerah saat menghadapi kendala
\end{itemize}

Ingat, TA bukan hanya tentang mendapatkan gelar, tapi juga tentang mengembangkan kemampuan berpikir kritis, problem-solving, dan research skills yang akan berguna sepanjang karir Anda di bidang Informatika.

\textbf{Selamat memulai perjalanan penelitian Anda!}