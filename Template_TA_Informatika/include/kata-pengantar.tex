\preface % Note: \preface JANGAN DIHAPUS!

Segala puji dan syukur kehadiran Allah SWT yang telah melimpahkan rahmat dan hidayah-Nya kepada kita semua, sehingga penulis dapat menyelesaikan penulisan tugas akhir yang berjudul \textbf{“Perbandingan Kinerja Arsitektur VGG19-LSTM dan BLIP dalam \textit{Visual Question Answering} (VQA) pada Citra Medis”} yang telah dapat diselesaikan sesuai rencana. Penulis banyak mendapatkan berbagai pengarahan, bimbingan, dan bantuan dari berbagai pihak. Oleh karena itu, melalui tulisan ini penulis mengucapkan rasa terima kasih kepada:

\begin{enumerate}
	\item Dewi Ratna Sari, SE.MM, dan Fadhli, SE.Ak, sebagai kedua orang tua penulis, senantiasa memberikan dukungan penuh terhadap aktivitas dan kegiatan yang dilakukan penulis, baik secara moral maupun material, serta menjadi motivasi terbesar bagi penulis untuk menyelesaikan tugas akhir ini.
    \item Prof. Dr. Taufik Fuadi Abidin, S.Si., M.Tech, sebagai Dekan Fakultas MIPA Universitas Syiah Kuala.
    \item Dr. Nizamuddin, M.Info.Sc., sebagai Ketua Jurusan Informatika Fakultas MIPA Universitas Syiah Kuala.
    \item Bapak Alim Misbullah, S.Si., M.S., selaku Dosen Pembimbing I dan Ketua Program Studi Informatika, telah memberikan bimbingan dan arahan yang berharga kepada penulis, membantu penulis dalam menyelesaikan Tugas Akhir ini.
    \item Ibu Laina Farsiah, S.Si., M.S., selaku Dosen Pembimbing II, memberikan bimbingan dan arahan yang sangat dibutuhkan oleh penulis, yang akhirnya membantu penulis menyelesaikan tugas akhir ini.
    \item Ibu Sri Azizah Nazhifah, S.Kom., M.Sc., selaku dosen pembahas, telah membantu penulis dan kawan-kawan untuk mempersiapkan ruang lingkup tugas akhir, serta memberikan arahan yang diperlukan.
    \item Bapak dr. Muhammad Ansari Adista, M.Pd.Ked., Sp.N., selaku dosen pembahas yang telah memberikan masukan dan saran yang sangat berharga bagi penulis dalam menyelesaikan tugas akhir ini.
    \item Ibu Dalila Husna Yunardi, B.Sc., M.Sc., selaku dosen wali, memberikan \textit{support} dan pedoman yang diperlukan selama penulis menempuh pendidikan.
    \item Muhhamad Razan Fawwaz, Amar Suhendra, Khairul Umam Albi, Muhhamad Rudy Hidayat, Muhhamad Raja Furqan, Teuku Nabil Muhhamad Dhuha, Yoan Rifqi Candra, dan Haris Daffa, telah menemani dan memberi dukungan kepada penulis selama empat tahun perkuliahan di jurusan Informaika USK.
	\item Anas Naufal Al-Kiram, Muhhamad Hanif, Daffa Mudhaffar, Muhhamad Ikhsan Fikri, dan Teuku Muhhamad Roy Adrian, teman yang selalu mengingatkan, dan membantu penulis untuk selalu berusaha dalam menyelesaikan tugas akhir di semester 8.
    \item Sahabat dan teman-teman seperjuangan dari Jurusan Informatika USK 2020 lainnya, telah memberikan dukungan moral yang luar biasa selama penulis menempuh pendidikan di jurusan Informatika USK.
    \item Seluruh Dosen dan Staf di Jurusan Informatika Fakultas MIPA, atas ilmu dan bimbingan yang diberikan kepada penulis selama perkuliahan, menjadi bagian tak terpisahkan dalam proses pembelajaran penulis.
\end{enumerate}

%\vspace{1.5cm}

Penulis mengakui adanya kekurangan dalam tulisan ini, baik dari aspek materi, metode, maupun bahasa yang digunakan. Oleh karena itu, penulis mengundang kritik dan saran yang konstruktif dari para pembaca untuk meningkatkan kualitas Tugas Akhir ini. Penulis berharap bahwa tulisan ini dapat memberikan manfaat bagi banyak orang dan berkontribusi pada kemajuan ilmu pengetahuan.

\vspace{1cm}


% \begin{tabular}{p{7.5cm}l}
% 	&Banda Aceh, 14 Januari 2024\\
% 	&\\
% 	&\\
% 	&\underline{Abdul Hafidh} \\
% 	NPM. 2008107010056
	
% \end{tabular}


% \begin{tabular}{p{7.5cm}l}
% 	&Banda Aceh, 8 Juli 2024\\
% 	&\\
% 	&\\
% 	&\multirow{1.5}{7.5cm}{\underline{Abdul Hafidh}} \\ 
% 	&NPM. 2008107010056 \\
% \end{tabular}
